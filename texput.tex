\title{Linux audio basics and \texttt{sounddevice}}

\maketitle

\begin{figure}
  \begin{center}
    \begin{tabular}{|c|c|}
      \hline
      \multicolumn{2}{c}{\href{https://en.wikipedia.org/wiki/List\_of\_Linux\_audio\_software}{Linux Audio Applications}} \\
      \hline
      \href{https://jackaudio.org/}{JACK} & \href{https://www.freedesktop.org/wiki/Software/PulseAudio/}{Pulse Audio} \\
      \hline
      \multicolumn{2}{c}{\href{https://www.alsa-project.org}{ALSA}} \\
      \hline
      \multicolumn{2}{c}{\href{https://www.alsa-project.org/wiki/Matrix:Main}{Sound hardware}} \\
      \hline
    \end{tabular}
  \end{center}
  \caption{The Linux Audio (problably partial) Stack.}
  \label{fig:stack}
\end{figure}

\href{https://www.freedesktop.org/wiki/Software/PulseAudio/}{PulseAudio
  is a sound server system for POSIX OSes, meaning that it is a proxy
  for your sound applications. It is an integral part of all relevant
  modern Linux distributions and is used in various mobile devices, by
  multiple vendors. It performs advanced operations on sound data as
  it passes between your application and hardware. Things like
  transferring audio to a different machine, changing the sample
  format or channel count, or mixing several sounds into one
  input/output, are easily achieved using
  PulseAudio.}~\cite{newmarch2017pulseaudio}

Like ALSA, PulseAudio is dinamically configured using
PulseAudio\href{https://www.freedesktop.org/wiki/Software/PulseAudio/Documentation/User/Modules/}{modules}. \texttt{\href{https://linux.die.net/man/1/pactl}{pactl}}
is the command line tool for loading and downloading modules. For
example, to use the equalizer
\texttt{\href{https://www.freedesktop.org/wiki/Software/PulseAudio/Documentation/User/Equalizer/}{qpaeq}}
that should be already installed in your computer if you are using
PulseAudio, run:

\begin{verbatim}
pactl load-module module-equalizer-sink
pulseaudio --kill && pulseaudio --start
qpaeq &
\end{verbatim}
% Ver también: pulseaudio-equalizer-ladspa

Another important PulseAudio application is the
\texttt{\href{https://freedesktop.org/software/pulseaudio/pavucontrol/}{pavcontrol}}
that is a mixer/VU-meter that handles sound applications, input and
output audio devices.

\section{Resources}

\bibliography{sound}
